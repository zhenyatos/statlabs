\documentclass[12pt,a4paper]{article}
\usepackage[utf8]{inputenc}
\usepackage[english,russian]{babel}
\usepackage{amssymb,amsfonts,amsmath,cite,enumerate,float,indentfirst}
\usepackage{graphicx}
\usepackage{geometry}
\usepackage{systeme}
\usepackage{hyperref}
\usepackage{url}
\usepackage[bottom]{footmisc}
\DeclareMathOperator{\med}{med}
\DeclareMathOperator{\sign}{sign}
\hypersetup{
	colorlinks,
	citecolor=black,
	filecolor=black,
	linkcolor=blue,
	urlcolor=black
}
\geometry{left=2cm}
\geometry{right=1.5cm}
\geometry{top=2cm}
\geometry{bottom=2cm}

\begin{document}
	
\begin{titlepage}
	\begin{center}		
		\vfill	
		Санкт-Петербургский политехнический университет \\
		Петра Великого\\
		\vskip 1cm
		Институт прикладной математики и механики \\
		Кафедра «Прикладная математика»
		\vfill
		\textbf{Отчёт\\
			по лабораторной работе №8\\
			по дисциплине\\
			«Математическая статистика»\\}
		\vfill
	\end{center}
	\vfill
	\hfill
	\begin{minipage}{0.4\textwidth}
		Выполнил студент:\\
		Самутичев Евгений Романович\\
		группа: 3630102/70201\\
	\end{minipage}
	\vfill
	\hfill 
	\begin{minipage}{0.4\textwidth}
		Проверил:\\
		к.ф.-м.н., доцент\\
		Баженов Александр Николаевич\
	\end{minipage}
	\vfill
	\begin{center}
		Санкт-Петербург\\2020 г.
	\end{center}
\end{titlepage}

\tableofcontents
\listoftables
\pagebreak

\section{Постановка задачи}
Для двух выборок размерами 20 и 100 элементов, сгенерированных согласно нормальному закону $N(0, 1)$, для параметров положения и масштаба построить асимптотически нормальные интервальные оценки на основе точечных оценок метода максимального правдоподобия и классические интервальные оценки на основе статистик $\chi^2$ и Стьюдента. В качестве доверительной вероятности взять $\gamma = 0.95$.
\pagebreak

\section{Теория}
\subsection{Интервальное оценивание}
\textit{Интервальной оценкой} (или \textit{доверительным интервалом}) числовой характеристики или параметра распределения $\theta$ генеральной совокупности с доверительной вероятностью $\gamma$ называется интервал $(\theta_1, \theta_2)$, границы которого являются случайными функциями: $\theta_1 = \theta_1 (x_1, ..., x_n), \theta_2 = \theta_2 (x_1, ..., x_n)$, который накрывает $\theta$ с вероятностью $\gamma$:
\begin{equation}
	\mathbf{P}(\theta_1 < \theta < \theta_2) = \gamma
\end{equation}

Часто вместо доверительной вероятности $\gamma$ рассматривается  \textit{уровень значимости} $\alpha = 1 - \gamma$. Важной характеристикой данной интервальной оценки является половина длины доверительного интервала, она называется \textit{точностью} интервального оценивания
\begin{equation}
	\Delta = \frac{\theta_2 - \theta_1}{2}
\end{equation}

\label{method}
Рассмотрим общий метод построения интервальных оценок \cite[стр. 456- -- 457]{verrazdely}. Пусть известна статистика $Y(\widehat{\theta}, \theta)$, содержащая оцениваемый параметр $\theta$ и его точечную оценку $\widehat{\theta}$ со следующими свойствами:
\begin{itemize}
	\item Функция распределения $F_Y (x)$ известна и не зависит от $\theta$
	\item Функция $Y (\widehat{\theta}, \theta)$ непрерывна и строго монотонна (для определенности - строго возрастает) по $\theta$
\end{itemize}
которые мы будем проверять при построении интервальных оценок нормального распределения. Зададим уровень значимости $\alpha$ и будем строить доверительный интервал так чтобы $(-\infty, \alpha_1), (\alpha_2, +\infty)$ накрывали $\theta$ с вероятностью $\frac{\alpha}{2}$. 

Пусть $y_{\alpha / 2}, y_{1 - \alpha / 2}$ -- квантили распределения $Y$ соотв. порядков, тогда
\begin{align}
\begin{split}
	\textbf{P}\left(y_{\alpha / 2}<Y(\widehat{\theta}, \theta)<y_{1 - \alpha / 2}\right)=F_{Y}\left(y_{1-\alpha / 2}\right)-F_{Y}\left(y_{\alpha / 2}\right) &=\\
	=1-\alpha / 2-\alpha / 2=1 &-\alpha=\gamma
\end{split}
\end{align}

Т.к. $Y(\widehat{\theta}, \theta)$ -- строго возрастает по $\theta$, то у неё есть обратная функция $Y^{-1}(y)$ относительно $\theta$ и она также строго возрастает, а значит:
\begin{align}
\begin{split}
	y_{\alpha / 2} &< Y(\widehat{\theta}, \theta) < y_{1 - \alpha / 2} \\
	Y^{-1}(y_{\alpha / 2}) &< \theta < Y^{-1}(y_{1 - \alpha / 2})
\end{split}
\end{align}
итого $\theta_1 = Y^{-1}(y_{\alpha / 2})$ и $\theta_2 = Y^{-1}(y_{1 - \alpha / 2})$ -- мы построили границы интервала. Применим это для построения интервальных оценок нормального распределения по выборке $(x_1, ..., x_n)$.

\subsection{Классические оценки}
\subsubsection{Для математического ожидания $m$}
Доказано что случайная величина $T = \sqrt{n-1} \cdot \frac{\bar{x} - m}{s}$ называемая статистикой Стьюдента, распределена по закону Стьюдента с $n-1$ степенями свободы, применяя с некоторыми деталями\cite[стр. 457 -- 458]{verrazdely} \hyperref[method]{выкладки}, получаем оценки границ интервала:
\begin{align}\label{lab8:1}
\begin{split}
	m_1 &= \bar{x} - \frac{x t_{1 - \alpha / 2} (n-1)}{\sqrt{n-1}} \\
	m_2 &= \bar{x} + \frac{x t_{1 - \alpha / 2} (n-1)}{\sqrt{n-1}}
\end{split}
\end{align}
, где $t_{1 - \alpha / 2}(n-1)$ -- квантиль порядка $1 - \alpha / 2$ распределения Стьюдента с $n-1$ степенями свободы.

\subsubsection{Для среднего квадратичного отклонения $\sigma$}
Доказано что случайная величина $n s^2 / \sigma^2$ распределена по закону $\chi^2$ с $n-1$ степенями свободы. Применяя общий \hyperref[method]{метод} построения интервальных оценок получаем оценки границ интервала:
\begin{align}\label{lab8:2}
\begin{split}
	\sigma_1 &= \frac{s \sqrt{n}}{\sqrt{\chi_{1-\alpha / 2}^{2}(n-1)}} \\
	\sigma_2 &= \frac{s \sqrt{n}}{\sqrt{\chi_{\alpha / 2}^{2}(n-1)}}
\end{split}
\end{align}
, где $\chi_{1 - \alpha / 2}^{2}(n-1), \chi_{\alpha / 2}^{2}(n-1)$ - квантили соотв. порядков $\chi^2$-распределения с $n-1$ степенями свободы.

\subsection{Асимптотически нормальные оценки}
\subsubsection{Для математического ожидания $m$}
В силу центральной предельной теоремы центрированная и нормированная случайная величина $\sqrt{n} (\bar{x} - m) / \sigma$ распределена приблизительно нормально с параметрами 0 и 1. Исходя из этого\cite[стр. 460]{verrazdely} получаем оценку:
\begin{align}\label{lab8:3}
\begin{split}
	m_1 &= \bar{x} - \frac{s u_{1 - \alpha / 2}}{\sqrt{n}} \\
	m_2 &= \bar{x} + \frac{s u_{1 - \alpha / 2}}{\sqrt{n}}
\end{split}
\end{align}
, где $u_{1 - \alpha /2}$ - квантиль нормального распределения $N(0, 1)$ порядка $1 - \alpha / 2$

\subsubsection{Для среднего квадратичного отклонения $\sigma$}
Аналогично, в силу центральной предельной теоремы центрированная и нормированная случайная величина $(s^2 - \mathbf{M}s^2) / \sqrt{\mathbf{D}s^2}$ при большом объеме выборки $n$ распределена приблизительно нормально с параметрами 0 и 1. Исходя из этого\cite[стр. 461]{verrazdely} получаем оценку:
\begin{align}\label{lab8:4}
\begin{split}
	\sigma_1 &= s\left(1 + u_{1 - \alpha / 2} \sqrt{(e+2) / n}\right)^{-1 / 2} \\
	\sigma_2 &= s\left(1 - u_{1 - \alpha / 2} \sqrt{(e+2) / n}\right)^{-1 / 2}
\end{split}
\end{align}
, где $e$ - выборочный эксцесс, определяемый как
\begin{equation}
	e = \frac{m_4}{s^4} - 3
\end{equation}
, где $m_4$ - четвертый выборочный центральный момент, определяемый как
\begin{equation}
	m_4 = \frac{1}{n} \sum_{i=1}^{n}\left(x_{i}-\bar{x}\right)^4
\end{equation}
\pagebreak

\section{Реализация}
\label{sec:impl}
Работа выполнена с использованием языка \textbf{Python} в интегрированной среде разработки \textbf{PyCharm}, были задействованы библиотеки:

\begin{itemize}
	\item \textbf{SciPy} - модуль \textbf{stats} для генерации данных, вычисления квантилей различных распределений для построения интервальных оценок
\end{itemize}

Исходный код работы приведен в \hyperref[sec:appl]{приложении}. 
\pagebreak

\section{Результаты}
\subsection{Классические оценки}
\begin{equation*}
\begin{array}{|c|c|c|}
	\hline 		& m\eqref{lab8:1} 	& \sigma\eqref{lab8:2} \\
	\hline n=20	& -0.6 < m < 0.27	& 0.71 < \sigma < 1.36 \\
	\hline n=100& -0.04 < m < 0.34 	& 0.84 < \sigma < 1.12 \\
	\hline
\end{array}
\end{equation*}

\subsection{Асимптотически нормальные оценки}
\begin{equation*}
\begin{array}{|c|c|c|}
	\hline 		& m\eqref{lab8:3} 	& \sigma\eqref{lab8:4} \\
	\hline n=20	& -0.56 < m < 0.23 	& 0.71 < \sigma < 1.46 \\
	\hline n=100& -0.03 < m < 0.34 	& 0.84 < \sigma < 1.13 \\
	\hline
\end{array}
\end{equation*}
\pagebreak

\section{Обсуждение}
Полученные интервальные оценки говорят о том что с вероятностью $0.95$ значения $m = 0$ и $\sigma = 1$ лежат в соответствующих интервалах. По постановке эксперимента, интервалы действительно накрывают истинные значения параметров. Следует заметить что при большом объеме $n$ выборки - асимптотические оценки практически совпадают с классическими.
\pagebreak

\section{Приложения}\label{sec:appl}
\noindent 1. Исходный код лабораторной {\url{https://github.com/zhenyatos/statlabs/tree/master/Lab8}}

\begin{thebibliography}{9} 
	\bibitem{verrazdely} \textbf{Вероятностные разделы математики.} Учебник для бакалавров технических направлений. // Под ред. Максимова Ю.Д. - СПб <<Иван Федоров>>, 2001. - 592 с., илл
\end{thebibliography}

\end{document}
