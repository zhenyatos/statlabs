\documentclass[12pt,a4paper]{article}
\usepackage[utf8]{inputenc}
\usepackage[english,russian]{babel}
\usepackage{amssymb,amsfonts,amsmath,cite,enumerate,float,indentfirst}
\usepackage{graphicx}
\usepackage{geometry}
\usepackage{systeme}
\usepackage{hyperref}
\usepackage{url}
\hypersetup{
	colorlinks,
	citecolor=black,
	filecolor=black,
	linkcolor=black,
	urlcolor=black
}
\geometry{left=2cm}
\geometry{right=1.5cm}
\geometry{top=2cm}
\geometry{bottom=2cm}

\begin{document}
	
\begin{titlepage}
	\begin{center}		
		\vfill	
		Санкт-Петербургский политехнический университет \\
		Петра Великого\\
		\vskip 1cm
		Институт прикладной математики и механики \\
		Кафедра «Прикладная математика»
		\vfill
		\textbf{Отчёт\\
			по лабораторной работе №1\\
			по дисциплине\\
			«Математическая статистика»\\}
		\vfill
	\end{center}
	\vfill
	\hfill
	\begin{minipage}{0.4\textwidth}
		Выполнил студент:\\
		Самутичев Евгений Романович\\
		группа: 3630102/70201\\
	\end{minipage}
	\vfill
	\hfill 
	\begin{minipage}{0.4\textwidth}
		Проверил:\\
		к.ф.-м.н., доцент\\
		Баженов Александр Николаевич\
	\end{minipage}
	\vfill
	\begin{center}
		Санкт-Петербург\\2020 г.
	\end{center}
\end{titlepage}

\tableofcontents
\listoffigures
\pagebreak

\section{Постановка задачи}
Для каждого из 5 распределений:

\begin{enumerate}
	\item Нормального $N(x, 0, 1)$
	\item Коши $C(x, 0, 1)$
	\item Лапласа $L(x, 0, \frac{1}{\sqrt{2}})$
	\item Пуассона $P(k, 10)$
	\item Равномерного $U(x, -\sqrt{3}, \sqrt{3})$	
\end{enumerate}

выборку размера: 10, 100, 1000 - сгенерировать 1000 раз, для каждой генерации произвести вычисления выборочных характеристик $\bar x, \text{med }x, z_R, z_Q, z_{tr}$ для всех генераций в рамках одного размера выборки получить значения среднего характеристик положения:
\begin{equation}
E(z) = \bar z
\end{equation}
и оценку дисперсии:
\begin{equation}
D(z) = \bar {z^2} - {\bar z}^2
\end{equation}
Представить полученные данные в виде таблиц.
\pagebreak

\section{Теория}
\subsection{Вариационный ряд}
Если элементы выборки $x_1, ..., x_n$ упорядочить по возрастанию на каждом элементарном исходе (рассматриваем их как случайные величины), получится новый набор случайный величин, называемый \textit{вариационным рядом}:
$$x_{(1)} \leq ... \leq x_{(n)}$$ Элемент $x_{(k)}$ называется \textit{k-ой порядковой статистикой}.

\subsection{Выборочные характеристики}
При работе с выборкой нам неизвестно распределение по которому она получена, а значит и соответствующие характеристики распределения. Однако, существуют оценки - т.н. \textit{выборочные характеристики}:

\begin{itemize}
	\item Выборочное среднее
	\begin{equation}
		\bar x = \frac{1}{n}\sum_{i=1}^{n}{x_i}
	\end{equation}
	
	\item Выборочная медиана
	\begin{equation}
		\text{med }x = 
		\begin{cases}
			x_{(k+1)} &\text{при $n=2k+1$}\\
			\frac{x_{(k)} + x_{(k+1)}}{2} &\text{при $n=2k$}
		\end{cases}
	\end{equation}
	
	\item Полусумма экстремальных выборочных элементов
	\begin{equation}
		z_R = \frac{x_{(1)} + x_{(n)}}{2}
	\end{equation}
	
	\item Выборочный квантиль уровня $\alpha$
	\begin{equation}
		z_{\alpha} = \frac{x_{(\lfloor q \rfloor+1)} +
							x_{(\lceil q \rceil+1)}}{2}, \text{где } q=(n-1)\alpha
	\end{equation}
	формула, используемая в \textbf{NumPy}, в этом случае $z_0 = \min\limits_{i=1,...,n}x_{(i)}, z_1 = \max\limits_{i=1,...,n}x_{(i)},
	\newline z_{0.5} = \text{med} \hspace{2pt} x$
	
	\item Полусумма квантилей
	\begin{equation}
		z_Q = \frac{z_{0.25} + z_{0.75}}{2}
	\end{equation}
	
	\item Усеченное среднее
	\begin{equation}
		z_{tr} = \frac{1}{n-2r}\sum_{i=r+1}^{n-r}x_{(i)}
	\end{equation}
\end{itemize}
	
	Выборочные характеристики являются случайными величинами, поэтому в работе и производится усреднение их значений для 1000 генераций и вычисление среднеквадратичного отклонения.

\pagebreak

\section{Реализация}
Работа выполнена с использованием языка \textbf{Python} в интегрированной среде разработки \textbf{PyCharm}, были задействованы библиотеки:

\begin{itemize}
	\item \textbf{NumPy} - построение вариационного ряда и вычисления
	\item \textbf{SciPy} - модуль \textbf{stats} для генерации данных по распределениям
\end{itemize}

Исходный код работы приведен в приложении. 
\pagebreak

\section{Результаты}

\pagebreak

\section{Обсуждение}

\pagebreak

\section{Приложения}
\noindent 1. Исходный код лабораторной {\url{https://github.com/zhenyatos/statlabs/tree/master/Lab2}}

\begin{thebibliography}{9} 
	\addcontentsline{toc}{section}{Список литературы}
	\bibitem{shiryaev} А. Н. Ширяев, \emph{Вероятность-1}. Изд. МЦНМО, Москва, 2017. 551 стр. 
	
	\bibitem{chernova} Н. И. Чернова, \emph{Математическая статистика: Учеб. пособие}. Новосиб. гос. ун-т. Новосибирск, 2007. 148 стр.
\end{thebibliography}

\end{document}
