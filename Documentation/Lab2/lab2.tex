\documentclass[12pt,a4paper]{article}
\usepackage[utf8]{inputenc}
\usepackage[english,russian]{babel}
\usepackage{amssymb,amsfonts,amsmath,cite,enumerate,float,indentfirst}
\usepackage{graphicx}
\usepackage{geometry}
\usepackage{systeme}
\usepackage{hyperref}
\usepackage{url}
\usepackage[bottom]{footmisc}
\hypersetup{
	colorlinks,
	citecolor=black,
	filecolor=black,
	linkcolor=black,
	urlcolor=black
}
\geometry{left=2cm}
\geometry{right=1.5cm}
\geometry{top=2cm}
\geometry{bottom=2cm}

\begin{document}
	
\begin{titlepage}
	\begin{center}		
		\vfill	
		Санкт-Петербургский политехнический университет \\
		Петра Великого\\
		\vskip 1cm
		Институт прикладной математики и механики \\
		Кафедра «Прикладная математика»
		\vfill
		\textbf{Отчёт\\
			по лабораторной работе №2\\
			по дисциплине\\
			«Математическая статистика»\\}
		\vfill
	\end{center}
	\vfill
	\hfill
	\begin{minipage}{0.4\textwidth}
		Выполнил студент:\\
		Самутичев Евгений Романович\\
		группа: 3630102/70201\\
	\end{minipage}
	\vfill
	\hfill 
	\begin{minipage}{0.4\textwidth}
		Проверил:\\
		к.ф.-м.н., доцент\\
		Баженов Александр Николаевич\
	\end{minipage}
	\vfill
	\begin{center}
		Санкт-Петербург\\2020 г.
	\end{center}
\end{titlepage}

\tableofcontents
\listoftables
\pagebreak

\section{Постановка задачи}
Для каждого из 5 распределений:

\begin{enumerate}
	\item Нормального $N(x, 0, 1)$
	\item Коши $C(x, 0, 1)$
	\item Лапласа $L(x, 0, \frac{1}{\sqrt{2}})$
	\item Пуассона $P(k, 10)$
	\item Равномерного $U(x, -\sqrt{3}, \sqrt{3})$	
\end{enumerate}

выборку размера: 10, 100, 1000 - сгенерировать 1000 раз, для каждой генерации произвести вычисления выборочных характеристик $\bar x, \text{med }x, z_R, z_Q, z_{tr}$ для всех генераций в рамках одного размера выборки получить значения среднего характеристик положения:
\begin{equation}\label{1}
E(z) = \bar z
\end{equation}
и дисперсию:
\begin{equation}\label{2}
D(z) = \bar {z^2} - {\bar z}^2
\end{equation}
Представить полученные данные в виде таблиц.
\pagebreak

\section{Теория}
\subsection{Вариационный ряд}
Если элементы выборки $x_1, ..., x_n$ упорядочить по возрастанию на каждом элементарном исходе (рассматриваем их как случайные величины), получится новый набор случайных величин, называемый \textit{вариационным рядом}:
$$x_{(1)} \leq ... \leq x_{(n)}$$ Элемент $x_{(k)}$ называется \textit{k-ой порядковой статистикой}
\footnote{\cite{chernova} стр. 10} .

\subsection{Выборочные характеристики}
При работе с выборкой нам неизвестно распределение по которому она получена, а значит и соответствующие характеристики распределения. Однако, существуют оценки - т.н. \textit{выборочные характеристики}:

\begin{itemize}
	\item Выборочное среднее
	\begin{equation}\label{3}
		\bar x = \frac{1}{n}\sum_{i=1}^{n}{x_i}
	\end{equation}
	
	\item Выборочная медиана
	\begin{equation}\label{4}
		\text{med }x = 
		\begin{cases}
			x_{(k+1)} &\text{при $n=2k+1$}\\
			\frac{x_{(k)} + x_{(k+1)}}{2} &\text{при $n=2k$}
		\end{cases}
	\end{equation}
	
	\item Полусумма экстремальных выборочных элементов
	\begin{equation}\label{5}
		z_R = \frac{x_{(1)} + x_{(n)}}{2}
	\end{equation}
	
	\item Выборочный квантиль уровня $\alpha$
	\begin{equation}
		z_{\alpha} = \frac{x_{(\lfloor q \rfloor+1)} +
							x_{(\lceil q \rceil+1)}}{2}, \text{где } q=(n-1)\alpha
	\end{equation}
	формула, используемая в \textbf{NumPy}, в этом случае $z_0 = \min\limits_{i=1,...,n}x_{(i)}, z_1 = \max\limits_{i=1,...,n}x_{(i)},
	\newline z_{0.5} = \text{med} \hspace{2pt} x$
	
	\item Полусумма квантилей
	\begin{equation}\label{7}
		z_Q = \frac{z_{0.25} + z_{0.75}}{2}
	\end{equation}
	
	\item Усеченное среднее
	\begin{equation}\label{8}
		z_{tr} = \frac{1}{n-2r}\sum_{i=r+1}^{n-r}x_{(i)}, \text{где } r=\lceil \frac{n}{4} \rceil
	\end{equation}
\end{itemize}
	
	Выборочные характеристики как борелевские функции от случайных величин (выборки) также являются случайными величинами, поэтому в работе и производится усреднение их значений для 1000 генераций и вычисление дисперсии.

\pagebreak

\section{Реализация}
Работа выполнена с использованием языка \textbf{Python} в интегрированной среде разработки \textbf{PyCharm}, были задействованы библиотеки:

\begin{itemize}
	\item \textbf{NumPy} - построение вариационного ряда и вычисления
	\item \textbf{SciPy} - модуль \textbf{stats} для генерации данных по распределениям
\end{itemize}

Исходный код работы приведен в приложении. 
\pagebreak

\section{Результаты}

\begin{table}[h!]
	\centering
	\begin{tabular}{|l|l|l|l|l|l|}
		\hline&$\bar x\text{	} \hyperref[3]{(3)}$ &$\text{med }x\text{	} \hyperref[4]{(4)}$  &$z_R\text{	} \hyperref[5]{(5)}$  &$z_Q\text{	} \hyperref[7]{(7)}$  &$z_{tr}\text{	} \hyperref[8]{(8)}$  \\ \hline
		$n=10$&&&&& \\ \hline$E(z)$&-0.0&-0.0&-0.0&-0.0&-0.1 \\ \hline$D(z)$&0.094467&0.130519&0.187448&0.105385&0.069492 \\ \hline
		
		$n=100$&&&&& \\ \hline$E(z)$&0.0&0.0&0.0&0.01&-0.01 \\ \hline$D(z)$&0.009651&0.015116&0.091466&0.011977&0.011309 \\ \hline
		
		$n=1000$&&&&& \\ \hline$E(z)$&-0.0&0.001&0.0&-0.0&-0.001 \\ \hline$D(z)$&0.001049&0.00153&0.062347&0.001313&0.001193 \\ \hline
		
	\end{tabular}
	\caption{Нормальное распределение}
\end{table}

\begin{table}[h!]
	\centering
	\begin{tabular}{|l|l|l|l|l|l|}
		\hline&$\bar x$ &$\text{med }x$  &$z_R$  &$z_Q$  &$z_{tr}$  \\ \hline
		$n=10$&&&&& \\ \hline$E(z)$&-0.0&-0.0&-1.0&-0.0&-0.2 \\ \hline$D(z)$&490.607293&0.332166&11914.643438&1.132547&0.210165 \\ \hline
		
		$n=100$&&&&& \\ \hline$E(z)$&2.0&0.0&79.0&0.0&-0.01 \\ \hline$D(z)$&3581.697241&0.026278&8922533.221739&0.04842&0.025421 \\ \hline
		
		$n=1000$&&&&& \\ \hline$E(z)$&1.0&0.0&642.0&0.0&-0.0 \\ \hline$D(z)$&2223.205146&0.00256&547218440.611133&0.004966&0.002619 \\ \hline
	\end{tabular}
	\label{tab:cauchy}
	\caption{Распределение Коши}
\end{table}

\begin{table}[h!]
	\centering
	\begin{tabular}{|l|l|l|l|l|l|}
		\hline&$\bar x$ &$\text{med }x$  &$z_R$  &$z_Q$  &$z_{tr}$  \\ \hline
		$n=10$&&&&& \\ \hline$E(z)$&0.0&0.0&0.0&0.0&-0.1 \\ \hline$D(z)$&0.098335&0.072722&0.382323&0.089897&0.041919 \\ \hline
		
		$n=100$&&&&& \\ \hline$E(z)$&-0.0&0.0&-0.0&-0.0&-0.01 \\ \hline$D(z)$&0.009719&0.00561&0.433362&0.009396&0.005806 \\ \hline
		
		$n=1000$&&&&& \\ \hline$E(z)$&0.0&0.001&-0.0&0.001&-0.0 \\ \hline$D(z)$&0.000918&0.000483&0.441511&0.000944&0.000561 \\ \hline
	\end{tabular}
	\caption{Распределение Лапласа}
\end{table}

\pagebreak

\begin{table}[h!]
	\centering
	\begin{tabular}{|l|l|l|l|l|l|}
		\hline&$\bar x$ &$\text{med }x$  &$z_R$  &$z_Q$  &$z_{tr}$  \\ \hline
		$n=10$&&&&& \\ \hline$E(z)$&10.0&10.0&10.0&10.0&7.0 \\ \hline$D(z)$&0.955624&1.3806&1.744716&1.128935&0.704541 \\ \hline
		
		$n=100$&&&&& \\ \hline$E(z)$&10.0&9.9&11.0&9.9&9.6 \\ \hline$D(z)$&0.097956&0.194391&0.997104&0.14328&0.110723 \\ \hline
		
		$n=1000$&&&&& \\ \hline$E(z)$&10.0&10.0&12.0&9.994&9.84 \\ \hline$D(z)$&0.010385&0.003484&0.6581&0.002748&0.011585 \\ \hline
	\end{tabular}
	\caption{Распределение Пуассона}
\end{table}

\begin{table}[h!]
	\centering
	\begin{tabular}{|l|l|l|l|l|l|}
		\hline&$\bar x$ &$\text{med }x$  &$z_R$  &$z_Q$  &$z_{tr}$  \\ \hline
		$n=10$&&&&& \\ \hline$E(z)$&0.0&0.0&-0.0&0.0&-0.1 \\ \hline$D(z)$&0.10033&0.234165&0.043909&0.136123&0.119729 \\ \hline
		
		$n=100$&&&&& \\ \hline$E(z)$&0.0&-0.0&0.001&0.0&-0.02 \\ \hline$D(z)$&0.009457&0.028559&0.00059&0.014028&0.018067 \\ \hline
		
		$n=1000$&&&&& \\ \hline$E(z)$&-0.001&-0.002&4e-05&-0.001&-0.003 \\ \hline$D(z)$&0.00102&0.003073&6e-06&0.001465&0.002005 \\ \hline
	\end{tabular}
	\caption{Равномерное распределение}
\end{table}

\pagebreak

\section{Обсуждение}
	\subsection{Математическое ожидание и медиана}
	Для каждого из указанных в постановке задачи распределений, приведем теоретические значения математического ожидания и медианы:
	
	\begin{itemize}
		\item $N(x, 0, 1): \mathbf{E}=0, \text{med}=0$ 
		\item $C(x, 0, 1): \mathbf{E} - \text{не определено}, \text{med}=0$
		\item $L(x, 0, \frac{1}{\sqrt{2}}): \mathbf{E}=0, \text{med}=0$
		\item $P(k, 10): \mathbf{E}=10, \text{med}=10$
		\item $U(x, -\sqrt{3}, \sqrt{3}): \mathbf{E}=0, \text{med}=0$
	\end{itemize}
	
	Как известно, \textit{выборочное среднее является несмещенной и состоятельной оценкой для математического ожидания}\footnote{\cite{chernova} стр. 17} Это объясняет то что для всех распределений кроме распределения Коши - выборочное среднее при росте $n$ стремится к математическому ожиданию, для распределения Коши последовательность вычислений не демонстрирует никакой сходимости (см. \hyperref[tab:cauchy]{таблицу 2}), поскольку у него отсутствует математическое ожидание. В тоже время медиана имеется у всех распределений и к ней сходится выборочная медиана.
	
	\subsection{Полусуммы: $z_R$ и $z_Q$}
	Полусумма квартилей $z_Q$ и экстремальных выборочных элементов $z_R$ оценивают центр симметрии распределения, из таблиц наблюдается что $z_Q$ ближе к медиане и последовательность вычислений $E(z) \text{ для } z_Q$ при увеличении $n$ сходится, в тоже время последовательность значений $E(z) \text{ для } z_R$ расходится при распределении Коши. Таким образом оценка через полусумму квартилей лучше, хотя и требует больше вычислений.
	
	\subsection{Упорядочение характеристик}
	Для $n=1000$ приведем упорядочение характеристик положения по каждому распределению:
	\begin{itemize}
		\item $N(x, 0, 1): z_{tr} < z_Q \leq \bar x \leq z_R < \text{med }x$
		\item $C(x, 0, 1): z_{tr} \leq z_Q \leq \text{med }x < \bar x < z_R$
		\item $L(x, 0, \frac{1}{\sqrt{2}}): z_{tr} \leq z_R \leq \bar x < \text{med }x \leq z_Q$
		\item $P(k, 10): z_{tr} < z_Q < \text{med }x \leq \bar x < z_R$
		\item $U(x, -\sqrt{3}, \sqrt{3}): z_{tr} < \text{med }x < z_Q \leq \bar x < z_R$
	\end{itemize}
\pagebreak

\section{Приложения}
\noindent 1. Исходный код лабораторной {\url{https://github.com/zhenyatos/statlabs/tree/master/Lab2}}

\begin{thebibliography}{9} 
	\bibitem{chernova} Н. И. Чернова, \emph{Математическая статистика: Учеб. пособие}. Новосиб. гос. ун-т. Новосибирск, 2007. 148 стр.
\end{thebibliography}

\end{document}
