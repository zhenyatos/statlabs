\documentclass[12pt,a4paper]{article}
\usepackage[utf8]{inputenc}
\usepackage[english,russian]{babel}
\usepackage{amssymb,amsfonts,amsmath,cite,enumerate,float,indentfirst}
\usepackage{graphicx}
\usepackage{geometry}
\usepackage{systeme}
\usepackage{hyperref}
\usepackage{url}
\usepackage[bottom]{footmisc}
\hypersetup{
	colorlinks,
	citecolor=black,
	filecolor=black,
	linkcolor=black,
	urlcolor=black
}
\geometry{left=2cm}
\geometry{right=1.5cm}
\geometry{top=2cm}
\geometry{bottom=2cm}

\begin{document}
	
\begin{titlepage}
	\begin{center}		
		\vfill	
		Санкт-Петербургский политехнический университет \\
		Петра Великого\\
		\vskip 1cm
		Институт прикладной математики и механики \\
		Кафедра «Прикладная математика»
		\vfill
		\textbf{Отчёт\\
			по лабораторной работе №2\\
			по дисциплине\\
			«Математическая статистика»\\}
		\vfill
	\end{center}
	\vfill
	\hfill
	\begin{minipage}{0.4\textwidth}
		Выполнил студент:\\
		Самутичев Евгений Романович\\
		группа: 3630102/70201\\
	\end{minipage}
	\vfill
	\hfill 
	\begin{minipage}{0.4\textwidth}
		Проверил:\\
		к.ф.-м.н., доцент\\
		Баженов Александр Николаевич\
	\end{minipage}
	\vfill
	\begin{center}
		Санкт-Петербург\\2020 г.
	\end{center}
\end{titlepage}

\tableofcontents
\listoftables
\pagebreak

\section{Постановка задачи}
Для каждого из 5 распределений:

\begin{enumerate}
	\item Нормального $N(x, 0, 1)$
	\item Коши $C(x, 0, 1)$
	\item Лапласа $L(x, 0, \frac{1}{\sqrt{2}})$
	\item Пуассона $P(k, 10)$
	\item Равномерного $U(x, -\sqrt{3}, \sqrt{3})$	
\end{enumerate}

выборку размера: 10, 100, 1000 - сгенерировать 1000 раз, для каждой генерации произвести вычисления выборочных характеристик $\bar x, \text{med }x, z_R, z_Q, z_{tr}$ для всех генераций в рамках одного размера выборки получить значения среднего характеристик положения:
\begin{equation}\label{1}
E(z) = \bar z
\end{equation}
и дисперсию:
\begin{equation}\label{2}
D(z) = \bar {z^2} - {\bar z}^2
\end{equation}
Представить полученные данные в виде таблиц.
\pagebreak

\section{Теория}
\subsection{Вариационный ряд}
Если элементы выборки $x_1, ..., x_n$ упорядочить по возрастанию на каждом элементарном исходе (рассматриваем их как случайные величины), получится новый набор случайных величин, называемый \textit{вариационным рядом}:
$$x_{(1)} \leq ... \leq x_{(n)}$$ Элемент $x_{(k)}$ называется \textit{k-ой порядковой статистикой}
\footnote{\cite{chernova} стр. 10} .

\subsection{Выборочные характеристики}
При работе с выборкой нам неизвестно распределение по которому она получена, а значит и соответствующие характеристики распределения. Однако, существуют оценки - т.н. \textit{выборочные характеристики}:

\begin{itemize}
	\item Выборочное среднее
	\begin{equation}\label{3}
		\bar x = \frac{1}{n}\sum_{i=1}^{n}{x_i}
	\end{equation}
	
	\item Выборочная медиана
	\begin{equation}\label{4}
		\text{med }x = 
		\begin{cases}
			x_{(k+1)} &\text{при $n=2k+1$}\\
			\frac{x_{(k)} + x_{(k+1)}}{2} &\text{при $n=2k$}
		\end{cases}
	\end{equation}
	
	\item Полусумма экстремальных выборочных элементов
	\begin{equation}\label{5}
		z_R = \frac{x_{(1)} + x_{(n)}}{2}
	\end{equation}
	
	\item Выборочный квантиль уровня $\alpha$
	\begin{equation}
		z_{\alpha} = \frac{x_{(\lfloor q \rfloor+1)} +
							x_{(\lceil q \rceil+1)}}{2}, \text{где } q=(n-1)\alpha
	\end{equation}
	формула, используемая в \textbf{NumPy}, в этом случае $z_0 = \min\limits_{i=1,...,n}x_{(i)}, z_1 = \max\limits_{i=1,...,n}x_{(i)},
	\newline z_{0.5} = \text{med} \hspace{2pt} x$
	
	\item Полусумма квантилей
	\begin{equation}\label{7}
		z_Q = \frac{z_{0.25} + z_{0.75}}{2}
	\end{equation}
	
	\item Усеченное среднее
	\begin{equation}\label{8}
		z_{tr} = \frac{1}{n-2r}\sum_{i=r+1}^{n-r}x_{(i)}, \text{где } r=\lceil \frac{n}{4} \rceil
	\end{equation}
\end{itemize}
	
	Выборочные характеристики как борелевские функции от случайных величин (выборки) также являются случайными величинами, поэтому в работе и производится усреднение их значений для 1000 генераций и вычисление дисперсии.

\pagebreak

\section{Реализация}
Работа выполнена с использованием языка \textbf{Python} в интегрированной среде разработки \textbf{PyCharm}, были задействованы библиотеки:

\begin{itemize}
	\item \textbf{NumPy} - построение вариационного ряда и вычисления
	\item \textbf{SciPy} - модуль \textbf{stats} для генерации данных по распределениям
\end{itemize}

Исходный код работы приведен в приложении. 
\pagebreak

\section{Результаты}

\begin{table}[h!]
	\centering
	\begin{tabular}{|l|l|l|l|l|l|}
		\hline&$\bar x\text{	} \hyperref[3]{(3)}$ &$\text{med }x\text{	} \hyperref[4]{(4)}$  &$z_R\text{	} \hyperref[5]{(5)}$  &$z_Q\text{	} \hyperref[6]{(6)}$  &$z_{tr}\text{	} \hyperref[8]{(8)}$  \\ \hline
		
		$n=10$&&&&& \\ \hline$E(z) \text{	} \hyperref[1]{(1)}$&0.012715&0.014862&0.015454&0.012652&-0.083313 \\ \hline$D(z) \text{	} \hyperref[2]{(2)}$&0.097193&0.125152&0.178521&0.111685&0.068423 \\ \hline
		
		$n=100$&&&&& \\ \hline$E(z)$&-0.000946&-0.006489&0.004489&-0.002539&-0.016852 \\ \hline$D(z)$&0.00996&0.015447&0.094226&0.012023&0.011609 \\ \hline
		
		$n=1000$&&&&& \\ \hline$E(z)$&-5.2e-05&-0.000798&0.008211&-9.9e-05&-0.001863 \\ \hline$D(z)$&0.001047&0.001693&0.059478&0.001248&0.001292 \\ \hline
		
	\end{tabular}
	\caption{Нормальное распределение}
\end{table}

\begin{table}[h!]
	\centering
	\begin{tabular}{|l|l|l|l|l|l|}
		\hline&$\bar x$ &$\text{med }x$  &$z_R$  &$z_Q$  &$z_{tr}$  \\ \hline
		$n=10$&&&&& \\ \hline$E(z)$&0.269175&0.003662&1.301074&0.035507&-0.148399 \\ \hline$D(z)$&420.187861&0.35542&10321.906112&1.007555&0.211997 \\ \hline
		
		$n=100$&&&&& \\ \hline$E(z)$&2.907996&0.003994&148.430856&-0.006257&-0.021858 \\ \hline$D(z)$&6161.9537&0.024654&15382087.074375&0.053289&0.026204 \\ \hline
		
		$n=1000$&&&&& \\ \hline$E(z)$&-0.388287&0.002631&-198.918484&0.00337&0.001214 \\ \hline$D(z)$&116.377519&0.0025&26681717.853264&0.004778&0.00254 \\ \hline
	\end{tabular}
	\label{tab:cauchy}
	\caption{Распределение Коши}
\end{table}

\begin{table}[h!]
	\centering
	\begin{tabular}{|l|l|l|l|l|l|}
		\hline&$\bar x$ &$\text{med }x$  &$z_R$  &$z_Q$  &$z_{tr}$  \\ \hline
		$n=10$&&&&& \\ \hline$E(z)$&0.006704&0.009902&0.009228&0.004885&-0.063547 \\ \hline$D(z)$&0.097983&0.077904&0.404708&0.087752&0.042383 \\ \hline
		
		$n=100$&&&&& \\ \hline$E(z)$&-0.001985&-0.001411&-0.016642&-0.000722&-0.011451 \\ \hline$D(z)$&0.009994&0.005558&0.383806&0.009967&0.005637 \\ \hline
		
		$n=1000$&&&&& \\ \hline$E(z)$&-0.000221&0.000461&-0.007886&0.000215&-0.000866 \\ \hline$D(z)$&0.001024&0.000529&0.425748&0.000971&0.000617 \\ \hline
	\end{tabular}
	\caption{Распределение Лапласа}
\end{table}

\pagebreak

\begin{table}[h!]
	\centering
	\begin{tabular}{|l|l|l|l|l|l|}
		\hline&$\bar x$ &$\text{med }x$  &$z_R$  &$z_Q$  &$z_{tr}$  \\ \hline
		$n=10$&&&&& \\ \hline$E(z)$&10.0066&9.8445&10.2815&9.93025&7.0925 \\ \hline$D(z)$&0.979396&1.41507&1.820008&1.116135&0.736944 \\ \hline
		
		$n=100$&&&&& \\ \hline$E(z)$&10.00372&9.867&10.924&9.9125&9.62518 \\ \hline$D(z)$&0.09631&0.197311&1.028224&0.14825&0.110399 \\ \hline
		
		$n=1000$&&&&& \\ \hline$E(z)$&9.997868&9.9985&11.654&9.994375&9.832432 \\ \hline$D(z)$&0.009556&0.001248&0.666284&0.003234&0.010793 \\ \hline
	\end{tabular}
	\caption{Распределение Пуассона}
\end{table}

\begin{table}[h!]
	\centering
	\begin{tabular}{|l|l|l|l|l|l|}
		\hline&$\bar x$ &$\text{med }x$  &$z_R$  &$z_Q$  &$z_{tr}$  \\ \hline
		$n=10$&&&&& \\ \hline$E(z)$&0.01133&0.017503&0.008043&0.011588&-0.105403 \\ \hline$D(z)$&0.097846&0.228939&0.042428&0.139057&0.118902 \\ \hline
		
		$n=100$&&&&& \\ \hline$E(z)$&-0.002344&-0.002894&0.001376&-0.002685&-0.020092 \\ \hline$D(z)$&0.009287&0.026082&0.000668&0.014104&0.017473 \\ \hline
		
		$n=1000$&&&&& \\ \hline$E(z)$&-0.001374&-0.003283&4.3e-05&-0.00104&-0.004076 \\ \hline$D(z)$&0.000993&0.00292&6e-06&0.001502&0.001978 \\ \hline
	\end{tabular}
	\caption{Равномерное распределение}
\end{table}

\pagebreak

\section{Обсуждение}
	\subsection{Математическое ожидание и медиана}
	Для каждого из указанных в постановке задачи распределений, приведем теоретические значения математического ожидания и медианы:
	
	\begin{itemize}
		\item $N(x, 0, 1): \mathbf{E}=0, \text{med}=0$ 
		\item $C(x, 0, 1): \mathbf{E} - \text{не определено}, \text{med}=0$
		\item $L(x, 0, \frac{1}{\sqrt{2}}): \mathbf{E}=0, \text{med}=0$
		\item $P(k, 10): \mathbf{E}=10, \text{med}=10$
		\item $U(x, -\sqrt{3}, \sqrt{3}): \mathbf{E}=0, \text{med}=0$
	\end{itemize}
	
	Как известно, \textit{выборочное среднее является несмещенной и состоятельной оценкой для математического ожидания}\footnote{\cite{chernova} стр. 17} Это объясняет то что для всех распределений кроме распределения Коши - выборочное среднее при росте $n$ стремится к математическому ожиданию, для распределения Коши последовательность вычислений не демонстрирует никакой сходимости (см. \hyperref[tab:cauchy]{таблицу 2}), поскольку у него отсутствует математическое ожидание. В тоже время медиана имеется у всех распределений и к ней сходится выборочная медиана.
	
	\subsection{Полусуммы: $z_R$ и $z_Q$}
	Полусумма квартилей $z_Q$ и экстремальных выборочных элементов $z_R$ оценивают центр симметрии распределения, из таблиц наблюдается что $z_Q$ ближе к медиане и последовательность вычислений $E(z) \text{ для } z_Q$ при увеличении $n$ сходится, в тоже время последовательность значений $E(z) \text{ для } z_R$ расходится при распределении Коши. Таким образом оценка через полусумму квартилей лучше, хотя и требует больше вычислений.
\pagebreak

\section{Приложения}
\noindent 1. Исходный код лабораторной {\url{https://github.com/zhenyatos/statlabs/tree/master/Lab2}}

\begin{thebibliography}{9} 
	\bibitem{chernova} Н. И. Чернова, \emph{Математическая статистика: Учеб. пособие}. Новосиб. гос. ун-т. Новосибирск, 2007. 148 стр.
\end{thebibliography}

\end{document}
