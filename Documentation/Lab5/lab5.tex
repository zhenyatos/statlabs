\documentclass[12pt,a4paper]{article}
\usepackage[utf8]{inputenc}
\usepackage[english,russian]{babel}
\usepackage{amssymb,amsfonts,amsmath,cite,enumerate,float,indentfirst}
\usepackage{graphicx}
\usepackage{geometry}
\usepackage{systeme}
\usepackage{hyperref}
\usepackage{url}
\usepackage[bottom]{footmisc}
\DeclareMathOperator{\med}{med}
\DeclareMathOperator{\sign}{sign}
\hypersetup{
	colorlinks,
	citecolor=black,
	filecolor=black,
	linkcolor=black,
	urlcolor=black
}
\geometry{left=2cm}
\geometry{right=1.5cm}
\geometry{top=2cm}
\geometry{bottom=2cm}

\begin{document}
	
\begin{titlepage}
	\begin{center}		
		\vfill	
		Санкт-Петербургский политехнический университет \\
		Петра Великого\\
		\vskip 1cm
		Институт прикладной математики и механики \\
		Кафедра «Прикладная математика»
		\vfill
		\textbf{Отчёт\\
			по лабораторной работе №5\\
			по дисциплине\\
			«Математическая статистика»\\}
		\vfill
	\end{center}
	\vfill
	\hfill
	\begin{minipage}{0.4\textwidth}
		Выполнил студент:\\
		Самутичев Евгений Романович\\
		группа: 3630102/70201\\
	\end{minipage}
	\vfill
	\hfill 
	\begin{minipage}{0.4\textwidth}
		Проверил:\\
		к.ф.-м.н., доцент\\
		Баженов Александр Николаевич\
	\end{minipage}
	\vfill
	\begin{center}
		Санкт-Петербург\\2020 г.
	\end{center}
\end{titlepage}

\tableofcontents
\listoffigures
\pagebreak

\section{Постановка задачи}
Сгенерировать двумерные выборки размера $20, 60, 100$ для нормального двумерного распределения $N(x, y, 0, 0, 1, 1, \rho)$.
Коэффициент корреляции $\rho$ взять равным $0, 0.5, 0.9$
Каждая выборка генерируется 1000 раз и для неё вычисляются: среднее значение, среднее значение квадрата и дисперсия коэффициентов
корреляции Пирсона, Спирмена и квадрантного коэффициента корреляции.

Повторить все вычисления для смеси нормальных распределений:
$$f(x,y) = 0.9\cdot N(x, y, 0, 0, 1, 1, 0.9) + 0.1\cdot N(x, y, 0, 0, 10, 10, -0.9)$$.
Изобразить сгенерированные точки на плоскости и нарисовать эллипс
равновероятности.
\pagebreak

\section{Теория}
\subsection{Двумерное нормальное распределение}
Двумерная случайная величина $(X,Y)$ называется \textit{распределенной нормально} (или просто \textit{нормальной}) если её плотность вероятности определена формулой

\begin{equation}
\resizebox{.9\hsize}{!}{$
	N(x, y, m_1, m_2, \sigma_1, \sigma_2, \rho) = \frac{1}{2\pi\sigma_1\sigma_2\sqrt{1-\rho^2}}
	\exp{
		\left(-\frac{1}{2(1-\rho^2)} 
		\left[\frac{(x - m_1)^2}{\sigma_1^2} -
		2\rho\frac{(x - m_1)(y - m_2)}{\sigma_1\sigma_2} +
		\frac{(y - m_2)^2}{\sigma_2^2}\right]\right)
	}$}\label{eq:1}
\end{equation}

Можно показать \cite[стр. 133-134]{verrazdely} что компоненты $X,Y$ двумерной нормальной случайной величины также распределены нормально с математическими ожиданиями $m_X = m_1, m_Y = m_2$ и среднеквадратическими отклонениями $\sigma_X = \sigma_1, \sigma_Y = \sigma_2$. В свою очередь параметр $\rho$ - называют \textit{коэффициентом корреляции}. Его значение будет раскрыто далее.

\subsection{Ковариация и коэффициент корреляции}
\textit{Ковариацией} двух случайных величин $X$ и $Y$ называется величина:
\begin{equation}
	K_{XY} = \textbf{M} \left[ (X-m_X)(Y-m_Y) \right]
\end{equation}
В свою очередь \textit{коэффициентом корреляции} называется
\begin{equation}
	\rho_{XY} = \frac{K_{XY}}{\sigma_X \sigma_Y}
\end{equation}

Коэффициент корреляции характеризует зависимость между случайными величинами $X$ и $Y$. Именно его мы задаем в двумерном нормальном распределении как $\rho$. Если случайные величины$X$ и $Y$  независимы, то $\rho_{XY} = 0$ т.к. в этом случае очевидно $K_{XY} = 0$.

\subsection{Выборочные коэффициенты корреляции}
\subsubsection{Пирсона}
Пусть по выборке значений $\{x_i, y_i\}_{i=1}^n$ двумерной случайной величины $(X,Y)$. Естественной оценкой для $\rho_{XY}$ служит \textit{выборочный коэффициент корреляции (Пирсона)}:
\begin{equation}\label{eq:4}
	r = \frac{\frac{1}{n} \sum_{i=1}^{n}{(x_i - \bar{x})(y_i - \bar{y})}}
	{\sqrt{
			\frac{1}{n} \sum_{i=1}^{n}{(x_i-\bar{x})^2}
			\frac{1}{n} \sum_{i=1}^{n}{(y_i-\bar{y})^2}}}
\end{equation}
Важным для приложений свойством является то что при данной оценке гипотеза $\rho_{XY} \neq 0$ (о наличии зависимости между случайными) величинами может быть принята на уровне значимости $0.05$ если выполнено: 
\begin{equation}\label{eq:5}
	|r|\sqrt{n-1} > 2.5
\end{equation}
это можно найти к примеру в \cite[стр. 538]{verrazdely}

\subsubsection{Квадрантный}
\textit{Выборочным квадрантным коэффициентом корреляции} называется величина:
\begin{equation}\label{eq:6}
	r_Q = \frac{(n_1 + n_3) - (n_2 + n_4)}{n}
\end{equation}
, где $n_1, n_2, n_3, n_4$ - количества элементов выборки попавших соответственно в I, II, III и IV квадранты декартовой системы координат с центром в $(\med{x}, \med{y})$ и осями \\
$x_1 = x - \med{x}, y_1 = y - \med{y}$, где $\med$ - выборочная медиана.

Формулу (\ref{eq:6}) можно переписать эквивалентным образом:
\begin{equation}\label{eq:7}
	r_Q = \frac{1}{n}\sum_{i=1}^{n}{\sign(x_i -\med{x})\sign(y_i -\med{y})}
\end{equation}
Важным свойством этой оценки является робастность. Её мы можем проверить используя схему засорения (смесь нормальных распределений).

\subsubsection{Спирмена}
На практике нередко требуется оценить степень взаимодействия между качественными признаками изучаемого объекта. Качественным называется признак, который нельзя измерить точно, но который позволяет сравнивать изучаемые объекты между собой и располагать их в порядке убывания или
возрастания их качества. Для этого объекты выстраиваются в определённом порядке в соответствии с рассматриваемым признаком. Процесс упорядочения называется ранжированием, и каждому члену упорядоченной последовательности объектов присваивается ранг, или порядковый номер. \\

Например, объекту с наименьшим значением признака присваивается ранг 1, следующему за ним объекту — ранг 2, и т.д. Таким образом, происходит сравнение каждого объекта со всеми объектами изучаемой выборки. Если объект обладает не одним, а двумя качественными признаками — переменными $X$ и $Y$, то для исследования их взаимосвязи используют выборочный коэффициент корреляции между двумя последовательностями рангов этих признаков. \\

Обозначим ранги, соотвествующие значениям переменной $X$, через $u$, а ранги, соотвествующие значениям переменной $Y$ - через $v$. \textit{Выборочный коэффициент ранговой корреляции Спирмена} определяется как выборочный коэффициент корреляции Пирсона между рангами $u$, $v$ переменных $X, Y$:
\begin{equation}\label{eq:8}
	r_S = \frac
			{\frac{1}{n} \sum_{i=1}^{n}{(u_i - \bar{u})(v_i - \bar{v})}}
			{\sqrt{\frac{1}{n} \sum_{i=1}^{n}{(u_i-\bar{u})^2}
					\frac{1}{n} \sum_{i=1}^{n}{(v_i-\bar{v})^2}}}
\end{equation}

\subsection{Эллипс равновероятности}
Рассмотрим выражение для плотности двумерного нормального распределения (\ref{eq:1}) несколько подробнее, а именно найдем линии уровня или что равносильно проекции сечения графика плотности плоскостями параллельными $xOy$ на плоскость $xOy$:
$$N(x, y, m_1, m_2, \sigma_1, \sigma_2, \rho) = const$$
, или что равносильно:
\begin{equation}\label{eq:9}
	\frac{(x - m_1)^2}{\sigma_1^2} - 2\rho\frac{(x - m_1)(y - m_2)}{\sigma_1\sigma_2} + \frac{(y - m_2)^2}{\sigma_2^2} = const
\end{equation}

Во всех точках каждого из таких эллипсов плотность двумерного нормального распределения $N(x, y, m_1, m_2, \sigma_1, \sigma_2, \rho)$ постоянна. Поэтому они и называются \textit{эллипсами равновероятности}\cite[стр. 44-45]{ventzel}. Отметим что в предельном случае $\rho = 1$:
$$\left(\frac{x-m_1}{\sigma_1} - \frac{y-m_2}{\sigma_2}\right)^2 = const$$
, такое уравнение задает семейство прямых паралелельных прямой:
\begin{equation}\label{eq:10}
	\frac{x-m_1}{\sigma_1} = \frac{y-m_2}{\sigma_2}
\end{equation}
Аналогично рассматривается предельный случай $\rho = -1$.

В данной работе, для выборки построенной по распределению $N(x, y, m_1, m_2, \sigma_1, \sigma_2, \rho)$ эллипсы равновероятности строились таким образом чтобы покрыть все элементы выборки т.е. в качестве константы, стоящей в правой части уравнения (\ref{eq:9}) бралась:
\begin{equation}
	R = \max_{\{(x_i, y_i)\}_{i=1}^n}{\left(\frac{(x_i - m_1)^2}{\sigma_1^2} - 2\rho\frac{(x_i - m_1)(y_i - m_2)}{\sigma_1\sigma_2} + \frac{(y_i - m_2)^2}{\sigma_2^2}\right)}
\end{equation}
\pagebreak

\section{Реализация}
\label{sec:impl}
Работа выполнена с использованием языка \textbf{Python} в интегрированной среде разработки \textbf{PyCharm}, были задействованы библиотеки:

\begin{itemize}
	\item \textbf{NumPy} - векторизация вычислений, работа с массивами данных, включая вычисление среднего и дисперсии
	\item \textbf{SciPy} - модуль \textbf{stats} для генерации данных по распределениям, вычисления коэффициентов корелляции
	\item \textbf{Matplotlib} - построение эллипсов рассеяния
\end{itemize}

Исходный код работы приведен в приложении. 
\pagebreak

\section{Результаты}
\subsection{Коэффициенты корреляции}
\begin{table}[h!]
	\centering
	\begin{tabular}{|l|l|l|l|}\hline $n=20$&$r$&$r_S$&$r_Q$  \\\hline$E(z)$&0.0&0.0&0.0\\\hline$E(z^2)$&0.1&0.1&0.1\\\hline$D(z)$&0.053556&0.053729&0.054264\\\hline &  &  &  \\\hline $n=60$&$r$&$r_S$&$r_Q$  \\\hline$E(z)$&0.0&0.0&0.0\\\hline$E(z^2)$&0.02&0.02&0.02\\\hline$D(z)$&0.016997&0.017097&0.018564\\\hline &  &  &  \\\hline $n=100$&$r$&$r_S$&$r_Q$  \\\hline$E(z)$&0.0&0.0&0.0\\\hline$E(z^2)$&0.01&0.01&0.01\\\hline$D(z)$&0.010095&0.010177&0.010416\\\hline &  &  &  \\\hline \end{tabular}

	\caption{$\rho = 0$}
\end{table}
\begin{table}[h!]
	\centering
	\begin{tabular}{|l|l|l|l|}\hline $n=20$&$r$&$r_S$&$r_Q$  \\\hline$E(z)$&0.5&0.5&0.3\\\hline$E(z^2)$&0.3&0.2&0.2\\\hline$D(z)$&0.03309&0.036427&0.046232\\\hline &  &  &  \\\hline $n=60$&$r$&$r_S$&$r_Q$  \\\hline$E(z)$&0.49&0.47&0.33\\\hline$E(z^2)$&0.25&0.23&0.12\\\hline$D(z)$&0.009512&0.010584&0.014097\\\hline &  &  &  \\\hline $n=100$&$r$&$r_S$&$r_Q$  \\\hline$E(z)$&0.5&0.48&0.33\\\hline$E(z^2)$&0.25&0.23&0.12\\\hline$D(z)$&0.006043&0.00652&0.008687\\\hline &  &  &  \\\hline \end{tabular}

	\caption{$\rho = 0.5$}
\end{table}
\pagebreak

\begin{table}[h!]
	\centering
	\begin{tabular}{|l|l|l|l|}\hline $n=20$&$r$&$r_S$&$r_Q$  \\\hline$E(z)$&0.891&0.86&0.7\\\hline$E(z^2)$&0.797&0.75&0.52\\\hline$D(z)$&0.002823&0.005097&0.02752\\\hline &  &  &  \\\hline $n=60$&$r$&$r_S$&$r_Q$  \\\hline$E(z)$&0.899&0.882&0.71\\\hline$E(z^2)$&0.809&0.78&0.51\\\hline$D(z)$&0.000641&0.001065&0.008925\\\hline &  &  &  \\\hline $n=100$&$r$&$r_S$&$r_Q$  \\\hline$E(z)$&0.898&0.885&0.71\\\hline$E(z^2)$&0.807&0.784&0.5\\\hline$D(z)$&0.000417&0.000637&0.004951\\\hline &  &  &  \\\hline \end{tabular}

	\caption{$\rho = 0.9$}
\end{table}

\begin{table}[h!]
	\centering
	\begin{tabular}{|l|l|l|l|}\hline $n=20$&$r$&$r_S$&$r_Q$  \\\hline$E(z)$&-0.0&0.5&0.5\\\hline$E(z^2)$&1.0&0.3&0.3\\\hline$D(z)$&0.448101&0.078135&0.039344\\\hline &  &  &  \\\hline $n=60$&$r$&$r_S$&$r_Q$  \\\hline$E(z)$&-0.6&0.48&0.56\\\hline$E(z^2)$&0.5&0.26&0.33\\\hline$D(z)$&0.079885&0.027009&0.01148\\\hline &  &  &  \\\hline $n=100$&$r$&$r_S$&$r_Q$  \\\hline$E(z)$&-0.7&0.47&0.56\\\hline$E(z^2)$&0.51&0.24&0.33\\\hline$D(z)$&0.029483&0.015814&0.006452\\\hline &  &  &  \\\hline \end{tabular}

	\caption{Смесь нормальных распределений}
\end{table}
\pagebreak

\subsection{Эллипсы равновероятности}
\begin{figure}[h!]
	\centering
	\includegraphics[scale=0.8]{ellipse/rho = 0.0, n = 20}
	\caption{$\rho = 0.0, n = 20$}
	\label{fig:image1}
\end{figure}
\begin{figure}[h!]
	\centering
	\includegraphics[scale=0.8]{ellipse/rho = 0.0, n = 60}
	\caption{$\rho = 0.0, n = 60$}
	\label{fig:image2}
\end{figure}
\pagebreak

\begin{figure}[h!]
	\centering
	\includegraphics[scale=0.8]{ellipse/rho = 0.0, n = 100}
	\caption{$\rho = 0.0, n = 100$}
	\label{fig:image3}
\end{figure}
\begin{figure}[h!]
	\centering
	\includegraphics[scale=0.8]{ellipse/rho = 0.5, n = 20}
	\caption{$\rho = 0.5, n = 20$}
	\label{fig:image4}
\end{figure}
\pagebreak

\begin{figure}[h!]
	\centering
	\includegraphics[scale=0.8]{ellipse/rho = 0.5, n = 60}
	\caption{$\rho = 0.5, n = 60$}
	\label{fig:image5}
\end{figure}
\begin{figure}[h!]
	\centering
	\includegraphics[scale=0.8]{ellipse/rho = 0.5, n = 100}
	\caption{$\rho = 0.5, n = 100$}
	\label{fig:image6}
\end{figure}
\pagebreak

\begin{figure}[h!]
	\centering
	\includegraphics[scale=0.8]{ellipse/rho = 0.9, n = 20}
	\caption{$\rho = 0.9, n = 20$}
	\label{fig:image7}
\end{figure}
\begin{figure}[h!]
	\centering
	\includegraphics[scale=0.8]{ellipse/rho = 0.9, n = 60}
	\caption{$\rho = 0.9, n = 60$}
	\label{fig:image8}
\end{figure}
\pagebreak

\begin{figure}[h!]
	\centering
	\includegraphics[scale=0.8]{ellipse/rho = 0.9, n = 100}
	\caption{$\rho = 0.9, n = 100$}
	\label{fig:image9}
\end{figure}
\pagebreak

\section{Обсуждение}
\subsection{Коэффициенты корреляции}
Для начала воспользуемся (\ref{eq:5}) для анализа экспериментов по которым были получены таблицы 1, 2. Выясним можно ли принять гипотезу о зависимости между случайными величинами на уровне значимости $\alpha = 0.05$ для $n = 100$ по коэффициенту Пирсона.
$$0 \sqrt{100 - 1} \leq 2.5, 4.98 \approx 0.5 \sqrt{100 - 1} > \cdot 2.5$$ 
В эксперименте 1 эту гипотезу принять нельзя, а в эксперименте 2 можно. При этом в эксперименте 1 с.в. заведомо независимы, а в эксперименте 2 зависимы, так что все согласуется с теорией. 

Из таблиц 1, 2 и 3 видно что $r, r_S$ являются состоятельными оценками $\rho_{XY}$ т.к. они все ближе к нему с ростом $n$.

Из таблицы 4 видим что $r_Q$ устойчивая к выбросам (робастная) оценка. Квадрантный коэффициент корреляции показывает лучшие результаты в устойчивости.

\subsection{Эллипсы равновероятности}
Видно что чем ближе $\rho$ к 1, тем эллипс равновероятности становится все больше похож на прямую, заданную как (\ref{eq:10}). Т.е. наглядно показано как между с.в. $X$ и $Y$ возникает линейная зависимость.
\pagebreak

\section{Приложения}
\noindent 1. Исходный код лабораторной {\url{https://github.com/zhenyatos/statlabs/tree/master/Lab5}}

\begin{thebibliography}{9} 
	\bibitem{verrazdely} \textbf{Вероятностные разделы математики.} Учебник для бакалавров технических направлений. // Под ред. Максимова Ю.Д. - СПб <<Иван Федоров>>, 2001. - 592 с., илл
	
	\bibitem{ventzel} Вентцель Е.С. \textit{Теория вероятностей: Учеб. для вузов.} — 6-е изд. стер. — М.: Высш. шк., 1999.— 576 c.
\end{thebibliography}

\end{document}
