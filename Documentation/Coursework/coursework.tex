\documentclass[12pt,a4paper]{article}
\usepackage[utf8]{inputenc}
\usepackage[english,russian]{babel}
\usepackage{amssymb,amsfonts,amsmath,cite,enumerate,float,indentfirst}
\usepackage{graphicx}
\usepackage{geometry}
\usepackage{systeme}
\usepackage{hyperref}
\usepackage{url}
\hypersetup{
	colorlinks,
	citecolor=black,
	filecolor=black,
	linkcolor=black,
	urlcolor=black
}
\geometry{left=2cm}
\geometry{right=1.5cm}
\geometry{top=2cm}
\geometry{bottom=2cm}

\begin{document}
	
\begin{titlepage}
	\begin{center}		
		\vfill	
		Санкт-Петербургский политехнический университет \\
		Петра Великого\\
		\vskip 1cm
		Институт прикладной математики и механики \\
		Кафедра «Прикладная математика»
		\vfill
		\textbf{Курсовая работа\\
			по дисциплине\\
			«Математическая статистика»\\}
		\vfill
	\end{center}
	\vfill
	\hfill
	\begin{minipage}{0.4\textwidth}
		Выполнил студент:\\
		Самутичев Евгений Романович\\
		группа: 3630102/70201\\
	\end{minipage}
	\vfill
	\hfill 
	\begin{minipage}{0.4\textwidth}
		Проверил:\\
		к.ф.-м.н., доцент\\
		Баженов Александр Николаевич\
	\end{minipage}
	\vfill
	\begin{center}
		Санкт-Петербург\\2020 г.
	\end{center}
\end{titlepage}

\tableofcontents
\listoffigures
\pagebreak

\section{Постановка задачи}
По данным выгруженным из .mat файла:
\begin{itemize}
	\item Изобразить движение проекции объекта на детектор в течение временного окна 161 с - 162 с (когда наблюдалось вращение)
	\item Выделить центр масс проекции
	\item Оценить скорость вращения центра масс
\end{itemize}
\pagebreak

\section{Теория}
\subsection{Оценка центра масс}
Обрабатываемые данные состоят из матриц интенсивности размера 16 х 16 т.е. мы работаем со значениями интенсивности $c_{ij}, i = 0, ..., 15, j = 0, ..., 15$. Сперва занулим значения интенсивности меньше среднего значения по всей матрице для того чтобы избавиться от сорных данных:
\begin{equation}
	\hat{c}_{ij} = 
	\begin{cases}
		0 &\text{, если } c_{ij} < \bar{c}  \\
		c_{ij} &\text{, иначе}
	\end{cases}
\end{equation}
, где $\bar{c}$ - выборочное среднее\cite{verrazdely} по выборке $\{c_{ij}\}_{0 \leq i,j \leq 15}$

Теперь вычислим координаты центра масс (в момент времени $t$ который задает матрицу интенсивности):
\begin{align}
\begin{split}
	x_t &= \sum_{i=0}^{15}{\sum_{j=0}^{15}{j\hat{c}_{ij}}} \\
	y_t &= \sum_{i=0}^{15}{\sum_{j=0}^{15}{i\hat{c}_{ij}}}
\end{split}
\end{align}

\subsection{Оценка скорости вращения}
Для оценки скорости вращенияво временной промежуток $[t, t + dt]$ (за который имеем две различных матрицы интенсивности) воспользуемся следующей оценкой модуля мгновенной скорости\cite{speed}:
\begin{equation}
	v_t = \frac{\|(x_{t+dt}, y_{t+dt}) - (x_t, y_t)\|_2}{dt}
\end{equation}
\pagebreak

\section{Реализация}
Работа выполнена с использованием языка \textbf{Python} в интегрированной среде разработки \textbf{PyCharm}, были задействованы библиотеки:

\begin{itemize}
	\item \textbf{NumPy} - работа с массивами данных
	\item \textbf{SciPy} - модуль \textbf{io} для обработки .mat файла 
	\item \textbf{OpenCV} - генерация видео по изображениям
	\item \textbf{Matplotlib} - генерация изображений
\end{itemize}
\pagebreak

\section{Результаты}
Приведем несколько изображений с выделением центра масс, полученных в ходе работы:
\begin{figure}[h!]
	\centering
	\includegraphics[scale=0.5]{1.png}
	\label{1}
\end{figure}

\begin{figure}[h!]
	\centering
	\includegraphics[scale=0.5]{2.png}
	\label{2}
\end{figure}

\begin{figure}[h!]
	\centering
	\includegraphics[scale=0.5]{3.png}
	\label{3}
\end{figure}
\pagebreak

\section{Обсуждение}
В ходе работы были получены оценки для скорости вращения из которых видно что оно неравномерное. Сгенерированное видео можно найти в \hyperref[appl]{Приложении}, по нему можно понять как менялась скорость объекта.
\pagebreak

\section{Приложения}\label{appl}
\noindent 1. Исходный код лабораторной {\url{https://github.com/zhenyatos/statlabs/tree/master/Coursework}}

\noindent 2. Видео
{\url{https://github.com/zhenyatos/statlabs/tree/master/Coursework/video.avi}}

\begin{thebibliography}{9} 
	\addcontentsline{toc}{section}{Список литературы}
	
	\bibitem{verrazdely} \textbf{Вероятностные разделы математики.} Учебник для бакалавров технических направлений. // Под ред. Максимова Ю.Д. - СПб <<Иван Федоров>>, 2001. - 592 с., илл
	
	\bibitem{speed} Кинематика точки // Википедия. [2020]. Дата обновления: 01.02.2020. URL: https://ru.wikipedia.org/?oldid=104900931 (дата обращения: 01.02.2020).
\end{thebibliography}


\end{document}
